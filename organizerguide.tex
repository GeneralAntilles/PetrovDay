\documentclass{article}

\title{Petrov Day Organizer Guide (Outdoor Setting)}
\author{James Babcock (jimrandomh@gmail.com)}
\date{September 26}

\begin{document}

\maketitle

Content note: This event is designed to provoke existential terror and involves
staring into the abyss.

\section{Overview}

Petrov Day is a yearly event on September 26 commemorating the anniversary of
the Petrov incident, where a false alarm in the Soviet early warning system
nearly set off a nuclear war. The purpose of the ritual is to make catastrophic
and existential risk emotionally salient, by putting it into historical context
and providing positive and negative examples of how it has been handled. This
is not for the faint of heart and not for the uninitiated; it is aimed at those
who already know what catastrophic and existential risk is, have some
background knowledge of what those risks are, and believe (at least on an
abstract level) that preventing those risks from coming to pass is important.

\section{COVID-19 Advisory for Petrov Day 2020}

The original ceremony is designed for groups of 5-10 people and is typically
performed indoors at a table. For Petrov Day 2020, to better accommodate
COVID-19 social distancing guidelines, we have modified the ceremony to be
outdoors, and made equipment and ceremonial changes that hew to the intentions
of the original while following these guidelines.

Many municipalities limit groups to 10, so if you have a larger group, then
you should split into separate groups ideally out of earshot of each other. If
you have more people and don't have space to split up comfortably, you might
want to encourage some of them to also hold Petrov Day ceremonies at their own
locations.

If participants are carpooling to the ceremony location, recommend limiting to
two people per vehicle and driving with the windows down or cracked.

When you invite people to attend, you should also clearly specify the
schedule. In order to reduce viral risk, the usual practice of having a group
dinner is not recommended. Also, you should warn people not to arrive in the
middle. Here is a sample email you might use to announce that you are hosting
Petrov Day:

\begin{center}
	\parbox{4.5in}{
	Dear friends,\newline\newline
	On September 26th, 1983, the world was nearly destroyed by nuclear war.
	That day is Petrov Day, named for the man who averted it. I will be
	hosting a ritual commemorating the occasion, on September 26th at
	$\langle$ADDRESS$\rangle$. We will gather at 6:00pm and begin the ritual
	at sunset (approximately 7:18pm). It will last for about an hour; please do
	not arrive in the middle.\newline\newline
	In preparation, please bring a printed copy of the booklet
	($\langle$LINK$\rangle$) and a pen or pencil. Additionally, in keeping with
	COVID-19 social distancing guidelines, we ask that you wear a mask and keep
	6 feet of distance from anyone not in your household. These guidelines will
	be observed throughout the ceremony.\newline\newline
	Sincerely,\newline
	$\langle$YOUR NAME$\rangle$}
\end{center}

\section{Materials}
You will need:

\begin{itemize} \itemsep0pt \parskip0pt \parsep0pt
	\item Each person should bring a printout of this booklet
	\item Each person should bring their own mask
	\item Each person should bring their own pen or pencil
	\item An outdoor space to accommodate the group
	\item 8 lanterns (to contain candle flame inside a window that can be opened and closed)
	\item 8 candles and a lighter
	\item Long candle (taper) to transfer flame between candles
	\item Candle holder for the taper
	\item Candle snuffer or other tool to extinguish candles (since blowing out a candle through a mask is difficult)
	\item A deck of small index cards or a pad of post-it notes, and some pens
	\item A ballot box or other container to hold the papers and keep them from
blowing away
	\item A red flashlight or electric lantern
	\item A portable table, ideally made of metal, on which the lanterns will be placed
	\item A fire extinguisher close enough to retrieve if needed
\end{itemize}

To accommodate an outdoor ceremony, enclosed candle lanterns are recommended.
Use a taper to transfer flame between candles.

\begin{itemize} \itemsep0pt \parskip0pt \parsep0pt
	\item Candle lantern: https://www.amazon.com/dp/B000BS05XS/
	\item Candles: https://www.amazon.com/dp/B074KSJ9KR/
	\item Taper candle: https://www.amazon.com/dp/B07YDYWS3N/
	\item Taper candle holder: https://www.amazon.com/dp/B07BHJFS5P/
\end{itemize}

Be mindful of local fire ordinances and environmental conditions. Find a
suitably safe outdoor location, such as a paved area, beach, or riverbed.
Alternatively, you may choose to substitute electric lanterns instead, and
symbolically ``transfer the flame" by touching one lantern to another.

There are two versions of the program: one with the pages in order for
single-sided printing, the other with the pages rearranged to print two-sided
and fold in the middle. This PDF file is the single-sided printing version,
which is better for reading on a computer screen, but the two-sided version is
preferred for printing on paper. Print it, stack the pages, fold them in half,
and staple the spine. Staple down into the front cover from the outside, 1/2
cm from the fold, at the top, middle, and bottom. The two-sided version is at
LINK{}. % TODO: Substitute the correct link here

\end{document}
