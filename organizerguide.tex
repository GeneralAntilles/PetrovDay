\documentclass{article}

\title{Petrov Day Organizer Guide}
\author{James Babcock (jimrandomh@gmail.com)}
\date{September 26}

\begin{document}

\maketitle

Content note: This event is designed to provoke existential terror and involves
staring into the abyss.

\section{Overview}

Petrov Day is a yearly event on September 26 commemorating the anniversary of
the Petrov incident, where a false alarm in the Soviet early warning system
nearly set off a nuclear war. The purpose of the ritual is to make catastrophic
and existential risk emotionally salient, by putting it into historical context
and providing positive and negative examples of how it has been handled. This
is not for the faint of heart and not for the uninitiated; it is aimed at those
who already know what catastrophic and existential risk is, have some
background knowledge of what those risks are, and believe (at least on an
abstract level) that preventing those risks from coming to pass is important.

\section{How to Organize}

A special note on COVID-19: The original ceremony is designed for groups of
5-10 people and is typically performed indoors at a table. To better
accommodate social distancing guidelines for COVID-19, we have updated the ceremony
to be outdoors, and made equipment and ceremony changes that hew to the
intentions of the original while following social distancing guidelines.

Many municipalities limit groups to 10, so if you have a larger group, then
you should split into separate groups ideally out of earshot of each other. If
you have more people and don't have space to split up comfortably, you might
want to encourage some of them to also hold Petrov Day ceremonies at their own
locations.

If participants are carpooling to the ceremony location, recommend limiting to
two people per vehicle and driving with the windows down or cracked.

Running Petrov Day is pretty easy. You need to invite people, acquire a
few simple props, and print one copy of the program for each person. You don't
have to write or rehearse anything, and once things get started, you'll be
doing the same things as any other participant. You might want to read through
the program in advance, but this isn't required.

When you invite people to attend, you should also clearly specify the
schedule. In order to reduce viral risk, dinner followed by ceremony is not
recommended. Also, you should warn people not to arrive in the middle. Here is
a sample email you might use to announce that you are hosting Petrov Day:

\begin{center}
	\parbox{4.5in}{
	Dear friends,\newline
	On September 26th, 1983, the world was nearly destroyed by nuclear war.
	That day is Petrov Day, named for the man who averted it. I will be
	hosting a ritual commemorating the occasion, on September 26th at
	$\langle$ADDRESS$\rangle$. We will gather at 6:00pm and begin the ritual
	at sunset (aproximately 7:18pm). It will last for about an hour; please do
	not arrive in the middle.\newline
	Sincerely,\newline
	$\langle$YOUR NAME$\rangle$}
\end{center}

\section{Materials}
You will need:

\begin{itemize} \itemsep0pt \parskip0pt \parsep0pt
    \item Each person should bring a printout of this booklet
    \item Each person should bring their own mask
    \item Each person should bring their own pen or pencil
    \item An outdoor space to accommodate the group
    \item 8 lanterns (to contain candle flame)
	\item 8 candles and a lighter
    \item Long candle (taper) to transfer flame between candles
    \item Candle holder for the taper
    \item Candle snuffer or other tool to extinguish candles
	\item A fire extinguisher close enough to retrieve if needed
	\item A deck of small index cards or a pad of post-it notes, and some pens
	\item A ballot box or other container to hold the papers and keep them from
blowing away
\end{itemize}

The candle-holder must hold at least eight candles. A Menorah will work, but it
shouldn't have symbols or iconography from Hanukkah or any other holiday. You
might want to put down aluminum foil to catch dripping wax. Also, you want
candles that won't burn too fast.

\begin{itemize} \itemsep0pt \parskip0pt \parsep0pt
	\item Candle holder: http://www.amazon.com/dp/B000BWPESK
	\item Candles: http://www.amazon.com/gp/product/B003U6ZVHS
\end{itemize}

There are two versions of the program: one with the pages in order for
single-sided printing, the other with the pages rearranged to print two-sided
and fold in the middle. This PDF file is the single-sided printing version,
which is better for reading on a computer screen, but the two-sided version is
preferred for printing on paper. Print it, stack the pages, fold them in half,
and staple the spine. Staple down into the front cover from the outside, 1/2
cm from the fold, at the top, middle, and bottom. The two-sided version is at
http://petrovday.com/downloads/PetrovDay-DoubleSidedBooklet.pdf{}.

\end{document}
